\documentclass[bachelor, och, pract]{SCWorks}
% параметр - тип обучения - одно из значений:
%    spec     - специальность
%    bachelor - бакалавриат (по умолчанию)
%    master   - магистратура
% параметр - форма обучения - одно из значений:
%    och   - очное (по умолчанию)
%    zaoch - заочное

% параметр - тип работы - одно из значений:
%    referat    - реферат
%    coursework - курсовая работа (по умолчанию)
%    diploma    - дипломная работа
%    pract      - отчет по практике
% параметр - включение шрифта
%    times    - включение шрифта Times New Roman (если установлен)
%               по умолчанию выключен
\usepackage[T2A]{fontenc}
\usepackage[utf8]{inputenc}
\usepackage{graphicx}

\usepackage[sort,compress]{cite}
\usepackage{amsmath}
\usepackage{amssymb}
\usepackage{amsthm}
\usepackage{fancyvrb}
\usepackage{longtable}
\usepackage{array}
\usepackage[english,russian]{babel}

\usepackage{verbatim,fancyvrb}

\usepackage{listings}
\usepackage{subcaption}

\usepackage[colorlinks=false]{hyperref}

\lstset{
	language=Python,
	numbers=left,               
    stepnumber=1,              
    numberfirstline=false,
    breaklines=true,
    basicstyle=\small
}

\begin{document}

% Кафедра (в родительном падеже)
\chair{математической кибернетики и компьютерных наук}

% Тема работы
\title{Супер название работы}

% Курс
\course{4}

% Группа
\group{451}

% Факультет (в родительном падеже) (по умолчанию "факультета КНиИТ")
%\department{механико"=математического факультета}

% Специальность/направление код - наименование
%\napravlenie{010300 "--- Фундаментальная информатика и информационные технологии}
%\napravlenie{010500 "--- Математическое обеспечение и администрирование информационных систем}
%\napravlenie{230100 "--- Информатика и вычислительная техника}
\napravlenie{09.03.04 "--- Программная инженерия}
%\napravlenie{090301 "--- Компьютерная безопасность}

% Для студентки. Для работы студента следующая команда не нужна.
%\studenttitle{Студентки}

% Фамилия, имя, отчество студента(ки) в родительном падеже
\author{Иванова Ивана Ивановича}

% Заведующий кафедрой
\chtitle{к.ф.~м.н., доцент} % степень, звание
\chname{С.~В.~Сидоров}  % Инициалы и фамилия

%Научный руководитель (для реферата преподаватель проверяющий работу)
\satitle{к.ф.-м.н., доцент} %должность, степень, звание
\saname{С.~В.~Сидоров}  % Инициалы и фамилия

% Руководитель практики от организации (только для практики,
% для остальных типов работ не используется)
\patitle{доцент, к.ф.-м.н.} %должность, степень, звание
\paname{В.~Г.~Сидоров}  % Инициалы и фамилия

% Семестр (только для практики, для остальных
% типов работ не используется)
\term{8} % номер

% Наименование практики (только для практики, для остальных
% типов работ не используется)
\practtype{преддипломная}

% Продолжительность практики (количество недель) (только для практики,
% для остальных типов работ не используется)
\duration{4} % число недель

% Даты начала и окончания практики (только для практики, для остальных
% типов работ не используется)
\practStart{06.05.2017}  % в формате ДД.ММ.ГГГГ
\practFinish{02.06.2017} % в формате ДД.ММ.ГГГГ

% Год выполнения отчета
\date{2017}  % в формате ГГГГ

\maketitle

% Включение нумерации рисунков, формул и таблиц по разделам
% (по умолчанию - нумерация сквозная)
% (допускается оба вида нумерации)
%\secNumbering

% \tableofcontents

% Раздел "Обозначения и сокращения". Может отсутствовать в работе
%\abbreviations

% Раздел "Определения". Может отсутствовать в работе
%\definitions


% Раздел "Определения, обозначения и сокращения". Может отсутствовать в работе.
% Если присутствует, то заменяет собой разделы "Обозначения и сокращения" и "Определения"
% \defabbr

% \intro
% В современном мире

% \conclusion
% В рамках данной 

% \begin{thebibliography}{9}
% \bibitem{} В.М.Вержбицкий, Основы численных методов: Учебник для вузов, Высш.шк. 2002.
% \end{thebibliography}

%Библиографический список, составленный с помощью BibTeX
%
\bibliographystyle{gost780uv}
% \bibliography{thesis}

% Окончание основного документа и начало приложений
% Каждая последующая секция документа будет являться приложением
% Приложения могут отсутствовать
% \appendix

\end{document}
